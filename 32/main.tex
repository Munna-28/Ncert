\iffalse
\let\negmedspace\undefined
\let\negthickspace\undefined
\documentclass[journal,12pt,twocolumn]{IEEEtran}
\usepackage{cite}
\usepackage{amsmath,amssymb,amsfonts,amsthm}
\usepackage{algorithmic}
\usepackage{graphicx}
\usepackage{textcomp}
\usepackage{xcolor}
\usepackage{txfonts}
\usepackage{listings}
\usepackage{enumitem}
\usepackage{mathtools}
\usepackage{gensymb}
\usepackage{comment}
\usepackage[breaklinks=true]{hyperref}
\usepackage{tkz-euclide} 
\usepackage{listings}
\usepackage{gvv}                                        
\def\inputGnumericTable{}                                 
\usepackage[latin1]{inputenc}                                
\usepackage{color}                                            
\usepackage{array}                                            
\usepackage{longtable}                                       
\usepackage{calc}                                             
\usepackage{multirow}                                         
\usepackage{hhline}                                           
\usepackage{ifthen}                                           
\usepackage{lscape}

\newtheorem{theorem}{Theorem}[section]
\newtheorem{problem}{Problem}
\newtheorem{proposition}{Proposition}[section]
\newtheorem{lemma}{Lemma}[section]
\newtheorem{corollary}[theorem]{Corollary}
\newtheorem{example}{Example}[section]
\newtheorem{definition}[problem]{Definition}
\newcommand{\BEQA}{\begin{eqnarray}}
\newcommand{\EEQA}{\end{eqnarray}}
\newcommand{\define}{\stackrel{\triangle}{=}}
\theoremstyle{remark}
\newtheorem{rem}{Remark}
\begin{document}

\bibliographystyle{IEEEtran}
\vspace{3cm}

\title{Probability Assignment}
\author{EE22BTECH11028-Katherapaka Nikhil$^{*}$% <-this % stops a space
}
\maketitle
\newpage
\bigskip
\renewcommand{\thefigure}{\theenumi}
\renewcommand{\thetable}{\theenumi}

Question:If X follows a binomial distribution with parameters n = 5, p and
$p_X(2) = 9p_X(3)$
then p is?\\
\fi
\solution
\begin{align}
p_X(2) &= \comb{5}{2} p^2 (1 - p)^{5 - 2}\\
&= \frac{5!}{2!3!} p^2 (1 - p)^3\\
&= 10 p^2 (1 - p)^3\\
p_X(3) &= \comb{5}{3} p^3 (1 - p)^{5 - 3}\\
&= \frac{5!}{3!2!} p^3 (1 - p)^2\\
&= 10 p^3 (1 - p)^2\\
9p_X(3) &= 9 \times 10 p^3 (1 - p)^2\\
&= 90 p^3 (1 - p)^2\\
\text{Given that } p_X(2) &= 9p_X(3)\\
\implies 10 p^2 (1 - p)^3 &= 90 p^3 (1 - p)^2\\
\implies (1 - p) &= 9p\\
\implies 10p &=1 \\
\implies p&=\frac{1}{10}
\end{align}
